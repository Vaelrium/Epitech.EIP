\documentclass[11pt]{report}

\usepackage[utf8]{inputenc}
\usepackage[french]{babel}
\usepackage{fullpage}
\usepackage{graphicx}
\usepackage{fancyhdr}	% headers/footers
\usepackage{xcolor}		% to use our own color
\usepackage{lastpage}	% to easily know the total number of pages
\usepackage{titling}	% to easily know the total number of pages
\usepackage{colortbl}	% to put color in a table background
\usepackage{datetime}	% to allow us set a new date formatting
\usepackage{multirow}   % to allow multirows in tables
\usepackage[colorlinks,linkcolor=black]{hyperref}
\usepackage{palatino}
%% \usepackage[colorlinks=false, urlcolor=blue, breaklinks, pagebackref, citebordercolor={0 0 0}, filebordercolor={0 0 0}, linkbordercolor={0 0 0}, pagebordercolor={0 0 0},
%%                      runbordercolor={0 0 0}, urlbordercolor={0 0 0}, pdfborder={0 0 0}]{hyperref}

% Custom defines zone

% Define useful hand-made colors
\definecolor{epiBlue}{RGB}{0,110,255}
\definecolor{lightGray}{gray}{0.92}

% Bit of code to bold an entire table row
% http://tex.stackexchange.com/questions/4811/make-first-row-of-table-all-bold
\newcolumntype{$}{>{\global\let\currentrowstyle\relax}}
\newcolumntype{^}{>{\currentrowstyle}}
\newcommand{\rowstyle}[1]{\gdef\currentrowstyle{#1}%
  #1\ignorespaces
}

% Defining a "dd/mm/yyyy" date format
\newdateformat{dashDate}{\twodigit{\THEDAY}/\twodigit{\THEMONTH}/\twodigit{\THEYEAR}}

% Define Document Title
\newcommand{\ProjectTitle}{Onitu}
\newcommand{\DocTitle}{Titre du document}
\newcommand{\SubTitle}{Sous-titre du document}

% Defining some logo image names
\newcommand{\ProjectLogo}{}
\newcommand{\EIPLogo}{logo_eip.png}

% Setting the space between each page's header and its content
\setlength{\headsep}{0.2in} 

% end of Defines


% fancyhdr-specific commands
\setlength{\headheight}{15.2pt}

%% Defining headers and footers contents.

% Big dirty hack of the "empty" pagestyle to show header and footer on the title page (in wait of a better solution)
\fancypagestyle{empty}
{
	\renewcommand{\headrulewidth}{0pt}
	\renewcommand{\footrulewidth}{1pt}
	\fancyhead[L]{\includegraphics[height=35pt]{\EIPLogo}}
	\fancyhead[C]{}
	\fancyhead[R]{\textcolor{epiBlue}{\textbf{\emph{\huge{\ProjectTitle}}}}}

	\fancyfoot[L]{}
	\fancyfoot[C]{\textcolor{epiBlue}{\textbf{\underline{\DocTitle\ — \SubTitle}}}}
	\fancyfoot[R]{}
}

\fancypagestyle{EIP}
{
	\renewcommand{\headrulewidth}{0pt}
	\renewcommand{\footrulewidth}{1pt}
	\fancyhead[L]{\includegraphics[height=35pt]{\EIPLogo}}
	\fancyhead[C]{}
	\fancyhead[R]{\textcolor{epiBlue}{\textbf{\emph{\huge{Onitu}}}}}

	\fancyfoot[L]{\textcolor{epiBlue}{\textbf{\underline{\leftmark}}}}
	\fancyfoot[C]{}
	\fancyfoot[R]{
		\thepage/\pageref{LastPage}
	}
}

\pagestyle{EIP} % does not seem to work ...

% end of fancyhdr stuff

%Gummi|063|=)

%\title{The Title\\\normalsize A Sub-title}
\title{
	\huge{\textbf{\textcolor{epiBlue}{\DocTitle} } }\\
	\Large{\textbf{\emph{\textcolor{gray}{\SubTitle} } } }
}


\begin{document}
\addtocontents{toc}{\protect\refstepcounter{page}} % makes the table of contents count pages from 1 (one)
\maketitle

\thispagestyle{empty}
\vspace*{10mm}

\textbf{\emph{\textcolor{onitu}{\large{Résumé du document} } } }\\

Résumé !!!

\clearpage


\thispagestyle{empty}
\vspace*{10mm}
\textbf{\emph{\textcolor{epiBlue}{\large{Description du documentstash stash} } } } \\

\vspace*{2mm}

\begin{tabular}{|>{\columncolor{epiBlue} \color{lightGray} \bfseries } l|l|}
\hline
	Titre & \DocTitle\\
\hline
	Date & \dashDate\today \\
\hline
	Auteur & Alexandre BARON\\
\hline
	Responsable & Louis Roché\\
\hline
	E-Mail & onitu\_2015@labeip.epitech.eu\\
\hline
	Sujet & Ceci n'est pas un template\\
\hline
	Mots clés & Mot, clé\\
\hline
	Version du modèle & 2.1\\
\hline
\end{tabular}

\vspace*{10mm}

\textbf{\emph{\textcolor{epiBlue}{\large{Tableau des révisions} } } }\\

\vspace*{2mm}

\begin{tabular}{|$l|p{4cm}|p{2cm}|p{5cm}|}
\hline
\rowcolor{epiBlue}
\rowstyle{ \color{lightGray} \bfseries}
	Date & \textcolor{lightGray}{\textbf{Auteur}} & \textcolor{lightGray}{\textbf{Section(s)}} & \textcolor{lightGray}{\textbf{Commentaires}}\\
\hline
	10/04/2013 & Alexandre Baron & Toutes & Nouveau template \\
\hline
	03/05/2013 & Alexandre Baron & Toutes & Lifting du template: Ajout de l'icône, changement des couleurs\\
\hline
	04/05/2013 & Alexandre Baron & Headers, Footers & Mis l'icone avec "Onitu", remonté l'icone EIP du footer au header \\
\hline
	& & & \\
\hline
\end{tabular}

\tableofcontents
\addtocontents{toc}{\protect\thispagestyle{empty}
                    \protect\pagestyle{empty}}
\thispagestyle{empty}

\chapter{Rappel de l'EIP}
\thispagestyle{EIP} % seems mandatory
\setcounter{page}{1} %reset the page count

\section{Qu'est-ce qu'un EIP et Epitech}
Epitech, école d'expertise informatique en cinq ans, offre aux étudiants l'opportunité de réaliser un projet de fin d'études sur trois ans, l'EIP (pour \emph{Epitech Innovative Project}).\\

À ce titre, les élèves doivent s'organiser en un groupe d'au moins cinq personnes et choisir un sujet porteur de nouveautés ou améliorant un ancien sujet. L'EIP est un passage obligatoire et unique dans la scolarité de l'étudiant, de par son envergure (18 mois) et la préparation requise. Le but est, à la fin du temps imparti, d'obtenir un projet commercialisable.


\section{Principe de base du système futur}
    Onitu est un projet visant à proposer une implémentation libre et Open Source du serveur d’Ubuntu One.\\

    Ubuntu One est un service de Canonical (sponsor officiel d'Ubuntu) permettant de disposer d’un espace de stockage en ligne qui sera synchronisé entre différents ordinateurs et périphériques compatibles via un logiciel client. Le client et le protocole d’Ubuntu One sont disponibles sous licence libre. Néanmoins, le serveur est propriétaire et n’a pas été publié.\\

    L'objectif d'Onitu de proposer un équivalent libre à ce serveur, afin de profiter des fonctionnalités d’Ubuntu One tout en maîtrisant le stockage des données et des informations.\\

    Les fichiers gérés par Onitu pourront être stockés sur un serveur administré par un utilisateur, ou bien sur des services tiers comme Dropbox, Amazon S3, ou Google Drive.\\

    La cible première d'Onitu est l'utilisateur averti, soucieux des problématiques de centralisation des données, et son entourage, à qui il fera profiter le serveur mis en place. Il n'est pas forcément technicien mais assez curieux, le profil même de l'utilisateur Ubuntu.
    Onitu vise aussi à être utilisé au sein d'entreprises ayant la volonté de maîtriser facilement le stockage de leur données.\\


\end{document}
