\chapter{Architecures, buts et contraintes}
\thispagestyle{EIP}

\section{Objectifs spécifiques ayant un impact sur l'architecure}

Le premier objectif est d'offrir une alternative libre au serveur
\emph{Ubuntu One}, et donc d'être entièrement compatible avec ce
dernier.

Aussi, pour une meilleure expérience utilisateur, cette solution se
devra d'être facilement déployable.

Un des objectifs est aussi de permettre de stocker les données sur des
services externes, tels \emph{Dropbox}, par l'intermédiaire de
\emph{drivers}. C'est principalement autour de cet objectif que se forme
l'architecture du projet.

\section{Contraintes fonctionnelles}

\begin{itemize}
\itemsep1pt\parskip0pt\parsep0pt
\item
  Système de queues de messages pour la communication entre
  \emph{drivers} -\textgreater{} core très basique (simplement les
  échanges de messages)
\item
  Réplication des données sur différents services
\item
  Configuration à l'aide d'un \emph{DSL}
\end{itemize}

\section{Contraintes non fonctionnelles}

\begin{itemize}
\itemsep1pt\parskip0pt\parsep0pt
\item
  Doit être conforme au protocole définit par \emph{Ubuntu One}
\item
  Utilisation de bibliothèques non portées en python3 (\emph{Twisted})
\item
  Le \emph{DSL} doit pouvoir être utilisé très simplement, pour ne pas
  rebuter l'utilisateur
\item
  La sécurité doit être maximale, c'est dans un soucis de protection des
  données que notre solution sera utilisée.
\end{itemize}