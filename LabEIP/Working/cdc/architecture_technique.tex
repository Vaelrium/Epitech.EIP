\section{Architecture technique}

Pour gérer le projet, l'ensemble du travail est regroupé sur une organisation github : \href{https://github.com/onitu/}{Onitu}.

Cette organisation permet notamment d'avoir un dépôt git pour chacune des parties du projet. Avec ces dépôts, un wiki et un bugtracker sont systématiquements disponibles. Ces outils sont utilisés pour organiser le projet. Pour toutes les étapes du développement, des milestones sont créées et des issues y sont liées. Cela permet de suivre le bon avancement du travail et de regrouper toutes les touches qui ont un lien.

Le wiki permet de mettre à disposition de tous la documentation de chacunes des parties, que ce soit la documentation technique à destination de développeurs ou de la documentation utilisateur pour installer ou configurer le logiciel.

Chaque partie du projet sera donc développée séparément des autres, mais un projet global regroupant tous les autres existera, tirant profit des submodules de git. Cela permet d'avoir plusieurs dépôts git dans un unique dépôt.

Tous les membres du groupe d'EIP ont les droits de commit sur l'ensemble des dépôts de l'organisation Onitu. Ils peuvent donc participer à chacunes des parties du projet, aussi bien sur du développement que sur de la documentation ou des rapports de bug.

Un outil de revue de code sera aussi utilisé pour faire valider chaque avancement, dans une optique de qualité et de sécurité.
