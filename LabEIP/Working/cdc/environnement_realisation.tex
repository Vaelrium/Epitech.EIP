\section{Environnement de réalisation}
Premièrement, un point important de la réalisation sera l'utilisation de Git pour l'hébergement du code, profitant d'un système de branches efficace et offrant ainsi à chacun un bon confort de développement. Ce Git étant hébergé sur Github, nous bénéficions de plus de fonctionnalités annexes telles que les issues ou le wiki de façon à disposer d'un espace clair pour discuter de différents points du projet, ou encore de rédiger de la documentation.\\

Nous n'imposerons aucun IDE, considérant que chacun est libre d'utiliser l'environnement avec lequel il sera le plus performant. Il va de paire que les fichiers de configuration liés à un tel IDE ou autres fichiers temporaires n'auront pas leur place sur le dépôt.\\

Le code sera écrit en python, langage interprété et dynamiquement typé, reconnu pour sa flexibilité et sa robustesse.\\
Ce code sera produit en respect de la PEP8, imposant diverses règles qui impliquent un code propre et agréable à la lecture, de plus très couramment usitée chez les développeurs python.\\

Aussi, nous serons dans le cadre de l'EIP soumis à des contraintes de portabilité. Ces contraintes seront annihilées par python, multi-plateforme par nature, de même que l'ensemble des biliothèques et technologies que nous envisageons d'utiliser.
