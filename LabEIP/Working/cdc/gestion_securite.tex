\section{Gestion de la sécurité}

La gestion de la sécurité est présente à plusieurs endroits dans le cycle de vie d'un logiciel. La première chose à laquelle on pense est de sécuriser le logiciel produit. C'est souvent par des attaques lors que le logiciel est en production que celui-ci est compromis. Mais ce n'est pas le seul risque. En effet il faut preter là même attention au procesus de dévelopement lui même, la gestion des diférentes ressources qui lui sont liées et à la facon dont le projet va être maintenu.

\subsection{Infrastructure \& maintenance}

Comme il est éxpliqué dans ce document, notre procesus de développement va s'organiser autour de Git et en particulier Github. Les membres du groupe d'EIP ont un acces administrateur sur cette platforme. Cela leur permet automatiquement de contribuer au code et d'utiliser les diferents outils fournis par Github. Git nous permet d'avoir une tracabilité des modifications du code car chaque modification est signé et authentifié.\\
Le code source est heberger sur les serveurs de Github, mais chaque developpeur à une copie du dépot en local à tout instant, donc une perte de donnée de la part de l'hébergeur n'est pas dramatique. Il n'y as pas non plus de problème de confidentialité car le projet est opensource et le code source n'est donc pas privée.\\
Vu que le projet est opensource il serais domage de se priver de contribution exterieures eventueles, pour cela là méthode clasique avec git est un pull-request. Il s'agit d'une demande d'intégration au dépot principal de code développé sur un autre dépot par une personne ne faisant pas partie du groupe d'EIP. Si un pull-request est fait, le code externe sera audité pour s'assurer de sa qualité, mais aussi pour vérifier que le code n'est pas malveillant, n'apporte pas de vulnérabilités et n'enleve rien à l'efficacité des solutions déjà en place.\\
Pour ce qui est de la maintenance et de la gestion des retours utilisateurs, ceux-ci pouront ce faire via le systeme de tickets fourni par Github. Tout le monde peut créer des tickets mais ceux si peuvent etre clos, suprimer ou administrer uniquement par les membres du groupe d'EIP.

\subsection{Processus de développement}

La quasi-totalité du projet sera développé en Python. Cella élimine une grande partie des vulnérabilités classiques que l'on rencontre dans des projets développer en C ou en C++. Notament la gestion de de là mémoire n'est plus à la charge du dévelopeur. De plus python lève des exeptions et arrete le programme dès qu'une valeur est mal utilisé. (overflows, casts incorect, ...)\\
Les interactions avec les bases de données et les differents protocoles seront faites a travers de librairies proposant un usage simple et moins propices aux erreurs qu'un acces direct aux ressources.\\
Cependant un certain nombres de vulnérabilités ne peuvent être éviter que part du bon sens et un respect des meilleures pratiques. C'est notament le cas pour tout ce qui concerne les fonctions qui permetent d'evaluer du code, ou d'acceder aux ressources systemes. L'appel a ces fonctions devra donc être particulierement surveiller et justifié, particulierement lors qu'ils peuvent être influencés par des entrées utilisateurs.\\
Nous organiseront des phases d'audit ou l'ensemble du code sera vérifié par des dévelopeurs n'ayant pas travaillés sur ces parties de facon à avoir un deuxieme avis objectif sur la question de la sécurité. De plus nous avons un partenariat avec le Laboratoire de Sécurité d'Epitech Toulouse qui poura effectuer un ou plusieurs audit exterieurs sur le code et les diferents protocoles utilisés.

\subsection{Sécurité dans Onitu}

En plus des diférentes mesures expliqué ci-dessus, une partie de la sécurité d'Onitu est apporté par ses composants principaux et les protocoles qui y sont utilisés.

\subsubsection{Ubuntu One}
Notre compatibilité avec les clients d'Ubuntu One nous impose leur protocole. Celui-ci à été concu et implémenté par des dévelopeurs profersionel et est utilisé en production depuis plusieurs années.\\
Comme UbuntuOne nous utiliserons SSL (Secure Socket Layer) et des certificats numériques pour assurer l'authentification du serveur et le chifrement des connections.\\
Pour authentifier les clients le protocole oauth est utilisé. Il s'agit d'un systeme d'authentification par token qui permet de gerer les acces parallelles depuis differents ordinateurs et éventuelement leur révocation.\\
Les fichiers stockés ne seront, eux, pas chifré car le serveur aura besoin de la version originale lors d'une demande de téléchargement. Cependant rien n'empeche l'utilisateur de chiffrer ses fichiers en local avant de les syncroniser avec le serveur Onitu.

\subsubsection{Backend drivers}
En ce qui concerne les diferent modules qui font géré le stockage de fichier sur le serveur Onitu, la sécurité de leur protocoles respectif ne dépend pas de nous. Nous ne pouront que conseiller l'utilisateur sur la configuration de la solution en fonction de ses attentes, (confidentialité, intégrité, authentification, ...) par exemple en conseillant ftps au lieu de ftp.
