\section{Technical architecture}
To handle the project, the work is centralised in the form of a Github organisation : \href{https://github.com/onitu/}{Onitu}.\\

The organisation allows us to maintain a git repository for each of the different components of the project. Each repository has its own bug tracker, which eases supervision of the project and allows for discutions.
For all the development steps \textit{milestones} are created and \textit{issues} are linked to those. This allows one to follow the work beeing done and groups everything that is linked to it.\\

Each repository can also have a wiki which can provide documentation for each of the parts, should it be the technical documentation for the developers or the user manuals need to install or configure the software.\\

All the parts of the project are developed separatly from one another. The conection bewteen them is made through \textit{submodules} of Git, which allow to integrate one git repository inside others.\\

All members of the EIP group have the rights to edit all the repositories of the Onitu organisation. This way they can activly participate in each of the project's components, working on the development, the documentations or bug repports.\\

The code-review tools that are integrated in github are used to validate each step, the goal beeing to increase quality and security.

