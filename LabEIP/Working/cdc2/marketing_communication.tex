\section{Comminucation et marketing}
La principale cible d'Onitu sont des utilisateurs avertis sensibles à la façon dont leurs données sont traitées. Onitu vise aussi les entreprises, elles aussi sensibles à la façon dont leurs données sont stockées. Ce n'est pas un public de masse et la façon de communiquer doit être choisie avec précaution.

\subsection{Communication}
Le support de communication principal d'Onitu est internet et plus précisement les sites spécialisés, parlant des nouvelles technologies. Il faudra convaincre ces sites d'écrire des articles sur le projet afin de le faire connaître.\\

Un des objectifs d'Onitu est d'être officiellement reconnu par Canonical, et donc de bénéficier de leurs supports de communication. Cela serait une opportunité très forte de faire connaître le projet.\\

\subsection{Marketing}
Le nom Onitu a été choisi afin d'évoquer Ubuntu One, tout en s'en démarquant. Pour le moment, aucun autre projet ne porte ce nom, et le référencement sur les moteurs de recherches ne devrait pas poser de problèmes.\\

La charte graphique du projet n'est pas encore définie. Elle devra décrire les couleurs associées au projet, ainsi que les éléments typographiques et un logotype.\\

Un site web sera mis en place afin de décrire le projet, de proposer un guide utilisateur, ainsi que des liens vers une documentation plus détaillée.
