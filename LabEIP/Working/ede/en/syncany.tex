\section{Syncany}
\thispagestyle{EIP}
\subsection{Présentation}
Syncany is a storage application that is meant to be multi-purpose. It provides the possibility to save your data on very differents supports like FTP, IMAP, SSH, CIFS, Amazon S3... The software must permit to manage ones files in a very flexible and personal way.

\subsection{Historique}
This project was launched on april 27th 2011. This project was very followed at the beggining, before loosing his attractivity.

In october 2011, this project became a student project at the Mannheim University. From this moment, its public activity strongly reduced. By now, there is almost no communication as the last official message is from april 2012.

\subsection{Description}
Syncany wants to be a thick crossplatform application which would allow secure storage (local encrypted data) and would abstract storage system. So it bears a large number of different storage ways, in a transparent maner :

\begin{itemize}
\renewcommand{\labelitemi}{$\bullet$}
\item Local Folder
\item FTP
\item IMAP
\item Google Storage
\item Amazon S3
\item Rackspace Cloud Files
\item WebDAV
\item Picasa Web Albums
\item Windows Share (NetBIOS/CIFS)
\item Box.net
\item SFTP/SSH
\end{itemize}

\vspace{1cm}

Users can with only one interface store their files in their medium of choice. It is a very flexible solution because you are not obliged to use the software itself to access, add and edit files. It allows to create various workflows, to ensure that the storage medium will always be available and even after Syncany ceases to exist, files will always be reachable.

\subsection{Critiques}
Even if Syncany´s offer seems to be very attractive, the leak of project-realted communication tends to asssume its death. And even if source code was available at the begining, it is not sufficent to be used in real situations.

As there is no official announcement of the project´s death, it is difficult to know if it is possible to contribute or even resume the work already done.

This solution is, by now, more a proof of concept than a ended software.
