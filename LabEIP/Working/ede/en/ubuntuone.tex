\section{Ubuntu One}
\thispagestyle{EIP} % seems mandatory
\subsection{Présentation}
Ubuntu One is a Dropbox-like software, with some multimedia functions. It's build by Canonical, the company which develop Ubuntu. It works with a closed server, but the client and the protocol are free.

Ubuntu One allows you to synchronize you files, share them, buy music and save your contacts

\subsection{Historique}
The first public release of Ubuntu One is in 2009. The software is include in Ubuntu since the version 9.10. In the beginning it was not a very popular software. But now it is a true success.

You can use it on many devices. The software is officially develop for: 
\begin{itemize}
\renewcommand{\labelitemi}{$\bullet$}
\item Ubuntu
\item Mac Os X
\item Windows
\item Iphone
\item Android
\end{itemize}

Ubuntu One is originaly more use by Ubuntu users. But now more of them are using windows.

To develop his product, Canonical had add some multimedia functions. It is possible to buy music online and the files will be send on the user account. You can also stream the music from your Ubuntu One account.

Is the protocol and the client are open source but there is not yet an open source server.

Canonical says that they have more than 1 000 000 customers since july 2011.

\subsection{Description}
The software is build in 3 parts: the client, the protocol and the server. The first and the second are free but not the server.

\subsubsection{Le client}
The code of the client is in majory build with python and Twisted for the network. The user interface is made with Qt. With the technologies the client can by run on many Operating systems.

A web interface can be use to look at your files.

\subsubsection{Le protocole}

The protocol of Ubuntu One was created specialy for the software.

It is build with protobuf, a library develop in google. It allows the developers to focus on the fonctionalities and to have a protocol which can be change. Many protobuf libraries exist for the most populars langages.

Canonical has open source an implementation of the protocol.

\subsubsection{Le serveur}

The server is the only closed source tool of Ubuntu One. But we know many things about it.

The data is saved in the Amazon S3 cloud. This is one reason why Canonical can not release on open source server.

For the authentification, Canonical is using OAuth. The web client is the only exception.

An REST API can be used to access to the public files fo the users.

\subsection{Critiques}

The biggest current lack of Ubuntu One is the closed source server. With Ubuntu One the user can not regain control of their data.

Compared to its competitors, the service also suffers from a smaller number of users. It is difficult to take advantage of the share functions.
