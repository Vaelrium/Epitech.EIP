Voici tout d'abord un récapitulatif sous forme tabulaire des projets évoqués dans ce document, et un relevé des différents points qui nous semblent importants pour une bonne expérience utilisateur.\\

\begin{tabular}{|$l|p{4cm}|p{4cm}|p{5cm}|}
\hline
\rowcolor{lightGray}
\rowstyle{ \color{epiBlue} \bfseries}
	Projet & Licence & Indépendant & Plate-formes \\
\hline
	Dropbox & Propriétaire & Non & GNU/Linux, Windows, Mac OS, Blackberry, iOS, Android \\
\hline
	Sparkleshare & GPL3 & Oui & GNU/Linux, Windows, Mac OS, iOS, Android \\
\hline
	Google drive & Propriétaire & Non & Web \\
\hline
	Skydrive & Propriétaire & Non & Windows, Mac OS, iOS, Web \\
\hline
	Owncloud & AGPL & Auto-hébergement & Toutes \\
\hline
	iCloud & Propriétaire & Non & Windows, Mac OS, iOS \\
\hline
	Syncany & GPL3 & Oui & Toutes \\
\hline
	Ubuntu One & Serveur propriétaire & Non & GNU/Linux, Windows, Mac OS, iOS, Android \\
\hline
	Onitu & Libre & Auto-hébergement & GNU/Linux, Windows, Mac OS, iOS, Android \\
\hline
\end{tabular}

\vspace*{10mm}

Notre projet vient donc renforcer Unbuntu One, dans la mesure où il permettra d'héberger le serveur de fichier où on le souhaite. Il hérite ainsi de ses avantages (client existant et multi-plateformes, protocole à base de protobuffers, etc.) tout en corrigeant son principal défaut, un serveur fermé et centralisé.

\vspace*{10mm}

SWOT