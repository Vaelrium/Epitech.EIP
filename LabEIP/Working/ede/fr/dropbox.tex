\section{Dropbox}
\thispagestyle{EIP} % seems mandatory
\subsection{Présentation}
DropBox est un service de stockage dans le cloud qui permet la synchronisation de fichiers entre différents terminaux. Dropbox a longtemps été la solution de référence et est énormément utilisé. Il est compatible avec GNU/Linux, Windows, Mac, Blackberrry, iOS et Android.\\

\subsection{Historique}
Le projet Dropbox est né au MIT en 2007 et a été lancé un an plus tard. En 2011, OPSWAT rapporte que Dropbox représente 14.14\% du marché mondial et cette même année, il dépasse les 50 millions d'utilisateurs. En 2012, ce chiffre est doublé et Dropbox annonce 100 millions d'utilisateurs.\\

\subsection{Description}
Dropbox crée un dossier spécial sur chaque ordinateur où il est installé. Il va synchroniser ce dossier entre les différents terminaux en répercutant les modifications apportées aux fichiers ou sous-dossiers. Les données placées dans ce dossier sont aussi accessibles depuis une interface web.\\

Les utilisateurs de Dropbox ont gratuitement accès à un espace de stockage de 18 Go, avec la possibilité de prendre un compte Pro pour bénéficier d'espace supplémentaire en payant un abonnement mensuel.\\

D'un point de vue technique, le serveur et le client Dropbox sont tous deux écrits en Python en utilisant des librairies standard telles que Twisted et ctypes. La gestion de l'historique d'un fichier est similaire à celle d'un gestionnaire de version classique dans le sens où il enregistre uniquement les différences entre deux versions successives d'un fichier, c'est le delta encoding.\\

Dropbox utilise le Amazon S3 pour ses serveurs.\\

\subsection{Critiques}

En juillet 2011, Dropbox a modifié ses conditions d'utilisation et peut maintenant utiliser les fichiers stockés par ses clients sans leur autorisation. Ceci a poussé de nombreux utilisateurs à abandonner le service et à s'orienter vers d'autres solutions ou l'auto-hébergement.\\
