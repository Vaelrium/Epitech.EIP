\section{Google drive}
\thispagestyle{EIP} % seems mandatory

\subsection{Présentation}
Google Drive est une application web de stockage dans le cloud. Il s'agit d'une solution propriétaire accessible depuis un site internet ou différentes applications spécifiques à certaines plate-formes.

\subsection{Historique}
Google Documents est une application web créé en 2006, regroupant un tableur, un traitement de textes et un éditeur de diaporamas.
Lancé en 2012, Google Drive lui succède, proposant en plus un système de stockage en ligne, valable pour un grand nombre de types de fichiers.\\

\subsection{Description}
Ce service hérite des fonctionnalités de Google Documents, telles que le partage de données avec d'autres utilisateurs, mais aussi, par son interface, une édition simultanée d'un même document par plusieurs personnes (travail collaboratif).\\
Le principal apport de Google Drive est certainement la possibilité de synchroniser ses données avec des fichiers locaux, qui fait de lui un véritable service de cloud computing.\\
Il s'intègre parfaitement à d'autres applications Google comme Google+ ou Gmail.\\
\\
Il offre des capacités allant de 5Go à 16To, tout en permettant 10Go maximum par fichier, et possède des applications spécifiques pour les plate-formes Windows, Mac OS, Android, et prochainement iOS.\\

\subsection{Critiques}
Derrière une interface intuitive et agréable à l'utilisation, on retrouve un problème assez embêtant: les données, centralisées, échappent à l'utilisateur. Ce dernier n'a en effet aucun contrôle sur ce qu'il stocke, et le code fermé de l'application empêche d'en porter une alternative sur un autre support.