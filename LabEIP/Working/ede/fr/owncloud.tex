\section{Owncloud}
\thispagestyle{EIP} % seems mandatory
\subsection{Présentation}
OwnCloud est une application web de stockage en ligne. Sous licence libre (AGPL), il peut-être installé sur n'importe quel serveur disposant de PHP et de SQL.\\
C'est une solution à installer soi-même, qui ne propose donc pas de louer un espace de stockage, mais d'en créer un.\\

\subsection{Historique}
Annoncé lors du Camp KDE 2010, le projet a bien évolué depuis. Le développement est très actif et suivi par la communauté (plus de 10000 commits et 1500 rapports de bogues en seulement 3 ans).\\

En 2011, une entreprise s'est créée autour du projet, proposant des services plus avancés pour les entreprises. Fin 2012, la société a réalisé une levée de fonds de 2,5 millions de dollars.

\subsection{Description}
OwnCloud repose sur un système d'applications. C'est donc un système modulaire, qui bénificie de très nombreuses fonctionnalités (édition de fichier, streaming de musique, synchronisation de calendrier et de contacts, galerie photo, etc…).\\
Plusieurs clients sont disponibles sur la plupart des plateformes, ainsi qu'une version web.\\

\subsection{Critiques}
Beaucoup d'avis d'utilisateurs critiquent les nombreux bogues d'OwnCloud, ainsi que la qualité de son code. Son plus gros avantage est qu'il permet d'installer rapidement une solution multi-usage, mais il semblerait qu'il se montre peu fiable sur le long terme.
Pour l'échange de fichiers, OwnCloud repose sur la technologie WebDAV, qui montre vite ses limites avec de gros fichiers.
À partir de la version 4.5, il est possible de s'interfacer avec Google Drive et Dropbox. Cette fonctionalité, encore expérimentale, et celle qui se rapproche le plus du Cloud Computing. En effet, OwnCloud ne permet pas de répartir la charge sur plusieurs serveurs, notion inhérente au Cloud Computing.
