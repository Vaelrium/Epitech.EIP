\chapter{Services de cloud par navigateur}
\thispagestyle{EIP} % seems mandatory

\section{Présentation}
Ces trois services sont assez similaires. notamment par rapport au fait qu'ils offrent une interface de gestion accessible depuis un navigateur internet, et sont donc compatibles tous systèmes. Cela ne les empêche pas de proposer des applications spécifiques pour différentes plate-formes.

\section{Google drive}

\subsection{Historique}
Google Documents est une application web créé en 2006, regroupant un tableur, un traitement de textes et un éditeur de diaporamas.
Lancé en 2012, Google Drive lui succède, proposant en plus un système de stockage en ligne, valable pour un grand nombre de types de fichiers.\\

\subsection{Description}
Ce service hérite des fonctionnalités de Google Documents, telles que le partage de données avec d'autres utilisateurs, mais aussi, par son interface, une édition simultanée d'un même document par plusieurs personnes (travail collaboratif).\\
Le principal apport de Google Drive est certainement la possibilité de synchroniser ses données avec des fichiers locaux, qui fait de lui un véritable service de cloud computing.\\
\\
Il offre des capacités allant de 5Go à 16To, tout en permettant 10Go maximum par fichier, et possède des applications spécifiques pour les plate-formes Windows, Mac OS, Android, et prochainement iOS.\\


\section{iCloud}

\subsection{Historique}
Créée en 2011, cette application de la firme Apple regroupe iTunes in the Cloud, Photo stream, Calendar, Mail, et Contacts.

\subsection{Description}
Il permet la synchronisation de différents éléments (applications, livres, documents, sauvegardes) entre divers appareils. Il ne s'agit pas à proprement parler d'un service par navigateur, mais il possède une interface web pour contrôler certaines de ses données.\\
DONNÉES BRUTES\\
Fonctionnalités\\
* L'interface web ne permet pas un contrôle total des fichiers (juste iWork et Photos)\\
* Le partage de fichiers n'est permis que via iWork\\
* Streaming de fichiers multimédia\\
* Intégré à iOS\\
* Backup automatique\\
\\
Capacités:\\
de 5Go à 55Go, 250Mo max / fichier uploadé\\
\\
Applications pour:\\
Windows, Mac, iOS\\


\section{Windows Live Skydrive}

\subsection{Historique}
Windows Live Folders est lancé puis ouvert au grand public en août 2007. Il change de nom le même mois pour devenir Window Live Skydrive. Offrant au départ un espace de stockage de 5Go par utilisateur, il passe à 25Go en 2008 pour redescendre à 7Go en 2012.\\

\subsection{Description}
Il peut s'utiliser aussi bien en tant que service web qu'en tant qu'application lourde, permettant une synchronisation avec les données locales.\\
DONNÉES BRUTES\\
Fonctionnalités\\
* Streaming de fichiers multimédia\\
* Permet de ne synchroniser que certains dossiers préalablement sélectionnés\\
* Travail collaboratif\\
* Intégration à Office et Windows Phone\\
* Possibilité d'accès à distance aux ordinateurs possédant le client\\
\\
Capacités:\\
de 7Go à 107Go, 2Go max / fichier uploadé\\
\\
Applications pour:\\
Windows Mac, Windows Phone, Android (prochainement), iOS\\

\section{Critiques}