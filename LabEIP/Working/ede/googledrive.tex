\chapter{Services de cloud par navigateur}
\thispagestyle{EIP} % seems mandatory

\section{Google drive}

\subsection{Présentation}
Google Documents est un service web créé en 2006, regroupant un tableur, un traitement de textes et un éditeur de diaporamas.
Lancé en 2012, Google Drive lui succède, proposant en plus un système de stockage en ligne, valable pour un grand nombre de types de fichiers.\\

\subsection{Description}
DONNÉES BRUTES\\
Fonctionnalités\\
* Ce système permet le partage de données avec d'autres membres, mais aussi, par son interface, une édition simultanée d'un même document par plusieurs personnes (travail collaboratif)\\
* Streaming de fichiers multimédia\\
* Permet de ne synchroniser que certains dossiers préalablement sélectionnés\\
* Intégration Gmail, G+\\
* Outil de recherche OCR\\
\\
Capacités:\\
de 5Go à 16 To, 10Go max / fichier uploadé\\
\\
Applications pour:\\
Windows, Mac, Android, iOS (prochainement)\\


\section{iCloud}

\subsection{Présentation}
Crée en 2011, cette application de la firme Apple regroupe iTunes in the Cloud, Photo stream, Calendar, Mail, et Contacts. Il permet la synchronisation de différents éléments (applications, livres, documents, sauvegardes) entre divers appareils. Il ne s'agit pas à proprement parler d'un service par navigateur, mais il possède une interface web pour contrôler certaines de ses données.\\

\subsection{Description}
DONNÉES BRUTES\\
Fonctionnalités\\
* L'interface web ne permet pas un contrôle total des fichiers (juste iWork et Photos)\\
* Le partage de fichiers n'est permis que via iWork\\
* Streaming de fichiers multimédia\\
* Intégré à iOS\\
* Backup automatique\\
\\
Capacités:\\
de 5Go à 55Go, 250Mo max / fichier uploadé\\
\\
Applications pour:\\
Windows, Mac, iOS\\


\section{Windows Live Skydrive}

\subsection{Présentation}
Il succède à Windows Live Folders distribué en 2007. Il peut s'utiliser aussi bien en tant que service web qu'en tant qu'application lourde, permettant une synchronisation avec les données locales.\\

\subsection{Description}
DONNÉES BRUTES\\
Fonctionnalités\\
* Streaming de fichiers multimédia\\
* Permet de ne synchroniser que certains dossiers préalablement sélectionnés\\
* Travail collaboratif\\
* Intégration à Office et Windows Phone\\
* Possibilité d'accès à distance aux ordinateurs possédant le client\\
\\
Capacités:\\
de 7Go à 107Go, 2Go max / fichier uploadé\\
\\
Applications pour:\\
Windows Mac, Windows Phone, Android (prochainement), iOS\\