\chapter{Sparkleshare}
\thispagestyle{EIP} % seems mandatory
\section{Présentation}
SparkleShare est une solution de stockage dans le cloud, orienté collaboration. Il repose sur Git pour gérer la synchronisation des fichiers. C'est un logiciel libre (GPLv3), compatible GNU/Linux, Windows, Mac, iOS et Android.\\

\section{Historique}
La version 1.0 est sortie le 9 décembre 2012 mais le travail a commencé début 2011 avec la première version publique le 14 fevrier 2011.\\

\section{Description}
Codé en C\#, SparkleShare repose essentiellement sur Git. Il est compatible avec n'importe quel dépôt Git, et peut donc être utilisé avec des services comme Github, Bitbucket ou Gitorious.\\

Son utilisation est essentiellement destinée aux utilisateurs qui voudraient travailler à plusieurs sur les mêmes fichiers, de manière plus conviviale qu'un outil de gestion de version classique.\\

\section{Critiques}
Aujourd'hui, peu d'utilisateurs semblent avoir adopté Sparkleshare, ce qui est probablement dû à son installation relativement complexe, ainsi qu'à son utilité limitée par rapport à des solutions comme Dropbox ou OwnCloud.\\

Git fonctionne très bien avec des fichiers textes, mais a plus de mal avec des binaires. Il est donc déconseillé d'utiliser Sparkleshare avec des fichiers multimédias.\\
Qui plus est, Sparkleshare ne propose pas de compatibilité avec les protocoles plus classiques, comme HTTP, FTP ou Webdav.\\