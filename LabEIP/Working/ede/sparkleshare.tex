\chapter{Sparkleshare}
\thispagestyle{EIP} % seems mandatory
\section{Presentation}
SparkleShare est une solution de stockage dans le cloud. Il est très proche de Dropbox dans le fonctionnement une fois configuré. Il repose sur Git pour gérer la synchronisation des fichiers. Ce logiciel est libre (sous licence GPL3), compatible GNU/Linux, Windows, Mac, iOS et Android. Il a la possibilité de se connecter à n'importe quel répertoire git dont github, gitorious, bitbucket.\\

\section{Historique}
La version 1.0 est sortie le 9 décembre 2012 mais le travail a commencé début 2011 avec la première version publique le 14 fevrier 2011.\\

\section{Description}
Le principal problème de ce logiciel est d'être basé sur git et donc de souffrir des mêmes défauts. Par exemple, sur de gros fichiers, git est lent. Par rapport à d'autres solutions, il n'est pas compatible avec d'autres protocoles (FTP/WebDAV/RSync pour ne citer qu'eux). Le logiciel ne fonctionne qu'avec un mode graphique, il est donc impossible d'utiliser le client en mode console. L'installation n'est pas des plus simples pour un utilisateur lambda qui veut utiliser son propre serveur.\\

D'un côté technologies utilisées, le logiciel se base sur git. Il est codé en C\# en utilisant mono.\\
