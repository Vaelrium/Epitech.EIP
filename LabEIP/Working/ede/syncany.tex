\chapter{Syncany}
\thispagestyle{EIP} % seems mandatory
\section{Présentation}
Syncany est une application de stockage qui veut polyvalente. Elle doit permettre de sauvegarder ses données sur des supports très variés, que ce soit FTP, IMAP, SSH, CIFS, Amazon S3… Le logiciel doit permettre de gérer de manière très souple et personnelle ses fichiers.

\section{Historique}
Le projet a été lancé le 27 avril 2011. Le projet était très suivi à ses débuts, avant de perdre peu en peu son attractivité. 

En octobre 2011, le projet est devenu un projet scolaire d'étudiants de University of Mannheim. A partir de ce moment, l'activité publique s'est fortement réduite. La communication est maintenant presque absente et le dernier message officiel date d'avril 2012.

\section{Description}
Syncany se veut être une application lourde multiplateforme qui permettrait un stockage sécurisé (données chiffrées localement) et faisant abstration du système de stockage. Elle supporte donc un très grand nombre de types de stockage différents de manière transparante :

\begin{itemize}
\renewcommand{\labelitemi}{$\bullet$}
\item Local Folder
\item FTP
\item IMAP
\item Google Storage
\item Amazon S3
\item Rackspace Cloud Files
\item WebDAV
\item Picasa Web Albums
\item Windows Share (NetBIOS/CIFS)
\item Box.net
\item SFTP/SSH
\end{itemize}

Le logiciel permet à travers d'une seule interface de stocker ses documents sur le support de son choix. C'est une solution très souple d'utilisation puisqu'elle n'impose pas l'utilisation du logiciel lui même pour accéder aux documents, en rajouter ou en modifier. Elle permet donc de créer des workflow très variés, de s'assurer que le support de stockage sera toujours disponible et que même après la disparition de syncany, les documents seront toujours accessibles.

\section{Critiques}
Si l'offre de syncany semblait des plus alléchantes, l'absence totale de communication autour du projet laisse aujourd'hui présumer de sa mort. Et même si le code source était disponible au début, il n'est pas dans un état suffisant pour une utilisation en condition réelles.

L'annonce officielle de mort du projet n'étant pas communiqué, il est aussi difficile de savoir s'il est possible de contribuer ou même reprendre le travail effectué.

La solution est à l'heure actuelle plus un proof of concept qu'un logiciel fini.
