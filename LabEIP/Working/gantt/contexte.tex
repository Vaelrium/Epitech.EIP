\section{Risques}
Notre projet étant dans le domaine du cloud qui est très à la mode ces dernières années, et qui voit naître de nouvelles solutions régulièrement, il y a un risque qu'une solution équivalent concurrente plus avancée voit le jour, mais nous supposons pour l'instant que ce ne sera pas le cas.

\section{Hypothèses}
Nous suposons que les developpeurs travaillant déjà sur les diférents systèmes par lesquels notre projet est concerné seront ouvert et disposé à nous aiguiller. D'après nos premiers échanges, nous supposons aussi qu'ils ne sont pas hostile à notre implémentation open-source.

Nous avons l'intention de ne développer qu'un serveur et donc de nous reposer sur des clients ubuntu-one déjà existants pour nos tests. Ces clients devront très probablement être modifiés légèrement de facon à pouvoir les faire fonctionner avec d'autres serveurs que ceux de canonical. Nous supposons que ces modification seront simple et bien recues par leur auteurs respectifs.

Nous ésperont pouvoir nous organiser de facon efficace durant la quatrième année en fesant des groupres de travail par fuseau horaire ce qui devrait augmenter notre productivité et tout de meme permettre une bonne avancée de projet durant l'année à venir.

\section{Contraintes}
Le projet étant open-source nous devrons respecter certaines contraintes imposés par les licenses choisies. Celles-ci doivent encore être étudiées et déterminées.

L'architecture complète du projet devra être fixé avant Juin 2013 pour pouvoir être presenté durant la Soutenance avec le LabEIP.

Une des contraintes principales de notre projet sera inhérente au dévelopement de celui-ci. En effet il faudra tenir compte du besoin de modification des clients et de l'implementation d'au moins certaines interfaces de stockage (par exemple le stockage local) avant de pouvoir réelement tester nos idées dans des conditions réelles.

Nous devrons aussi être capable d'organiser des réunions régulières, et ce malgrès le décalage horaire.

Enfin, du à l'éloigenement géographique il sera plus compliquer de maintenir une platforme de test pour notre solution d'hébergement. Nous seront probablement ammené a prendre un serveur dédiée de test accesible de nos diférentes destinations en tek4.

