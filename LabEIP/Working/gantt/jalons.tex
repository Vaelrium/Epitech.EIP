\subsection{Jalons du planning EIP Epitech}
Cet ensemble de jalons est commun à tous les groupes, car imposé par le planning EIP.


Une première famille de jalons est formée par les bilans d'architecture AA1 et AA2, respectivement du 20 juin et 1er septembre 2013. Ces jalons devront donc mettre terme aux choix et possibilités au niveau de l'architecture du projet.

Une seconde concerne les bilans techniques, et bilans techniques finaux, répartis de façon régulière de novembre 2013 à janvier 2015. Ils s'axeront avec des jalons propres à notre projet et devront marquer des tournants décisifs, tels que des finalisations de lots.


Enfin, la troisième rassemble les deux soutenances finales, de 4ème année en septembre 2014, et de 5ème en février 2015. Ils représenteront pour leur part des livraisons de versions avancée pour la première, et finale pour la seconde.


\subsection{Jalons propres au projet Onitu}
Ces jalons sont définis au niveau de notre projet et marquent donc des points importants de celui-ci. Ils sont à mettre en parallèle avec l'ensemble précédent, dans le sens où ces derniers représentaient des dates importantes, pour lesquelles ces points devront être atteints.


Tout d'abord, il sera nécessaire de posséder un client minimal de test, pour permettre le développement du serveur, tout en continuant celui du client.


Ensuite, l'obtention d'un serveur fonctionnel, puis d'un serveur complet et fonctionnel marqueront deux étapes très importantes du projet.


Les derniers jalons concernent les clients, et correspondent à la finalisation du fork du client officiel, et de la webUI.


\subsection{Définition du lotissement}
Notre lotissement est formé des différentes parties du projet, à savoir: le fork du client officiel Ubuntu One (ajoutant le choix du serveur à utiliser), la webUI (interface complète de gestion avec partage de fichiers), le serveur (conception et réalisation des API, de la base de données), et les drivers (Donnant accès à des ressources internes comme externes pour le stockage des données).