\section{Risks}

Our project being in the cloud domain, a very trendy subject since a few years, which regularly gives birth to new solutions, there is a risk that a concurrent equivalent, if not superior, solution comes to day, but we're assuming that it won't be the case for the moment.

\section{Hypothesis}

We're assuming that developers who are already working on other systems by which our project is concerned will be open-minded and disposed to point us to the right way. According to our first exchanges, we're supposing they aren't against our open source implementation.

We just want to develop a server and, thus, just lay ourselves on already existing Ubuntu One clients for our tests. These clients are very likely to be slightly modified in order to make them work with other servers than those provided by Canonical. We suppose these modifications will be simple and positively received by their authors.

We hope to be able to organize ourselves in an efficient way during our fourth year by making timezone workgroups, which should increase our productivity while permitting us a good development of the project during the year to come.

\section{Constraints}

The project being open source, we'll have to respect some constraints imposed by the licenses we'll choose. The latters have not been studied and chosen yet.

The complete project architecture shall be fixed before June 2013, in order to be presented during the defense with the LabEIP.

One of the main constraints of our project will be inherent to its development. Actually, we'll have to take account of the need to modify the clients and the implementation of at least some storage interfaces (local storage, for example) before being able to really test our ideas in real life conditions.

We'll also have to be able to organize regular meetings, in spite of the timezone difference.

Finally, due to the geographic distance, it will be harder to maintain a test platform for our hosting solution. We'll probably have to take a testing dedicated server available from all our Tech4 destinations.
