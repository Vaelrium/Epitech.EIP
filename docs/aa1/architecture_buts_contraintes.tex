\chapter{Architecture, buts et contraintes}
\thispagestyle{EIP} % seems mandatory

\section{Objectifs spécifiques ayant un impact sur l'architecture}

La réflexion sur l'architecture du projet passe d'abord par la
définition d'objectifs clairs, que voici énoncés.

Le premier objectif est d'offrir une alternative libre au serveur
\emph{Ubuntu One}, et donc d'être entièrement compatible avec ce
dernier.

Un autre est d'offrir à l'utilisateur un contrôle total sur ses données,
il lui revient de choisir où ses fichiers seront stockés.

Aussi, pour une meilleure expérience utilisateur, cette solution se
devra d'être facilement déployable.

Un des objectifs est aussi de permettre de stocker les données sur des
services externes, tels \emph{Dropbox}, par l'intermédiaire de
\emph{drivers}. C'est principalement autour de cet objectif que se forme
l'architecture du projet.

\section{Contraintes fonctionnelles}

Différentes contraintes permettent aussi de diriger les choix
techniques. Premièrement, les contraintes fonctionnelles, qui décrivent
les caractéristiques du système.

Il a été décidé que les différents \emph{drivers} communiqueraient
entre-eux à l'aide de queues de messages. Ainsi, le programme disposera
d'un \emph{core} très basique, ne s'occupant que de la transmission des
messages.

Une autre contrainte et la volonté de pouvoir assurer une réplication
des données sur différents services. Par l'intermédiaire des
\emph{drivers}, préalablement configurés, qui écouteront les
modifications d'un ensemble de fichiers pour les reproduire à
l'identique sur leur espace de stockage.

L'ensemble des modules devront pouvoir être configurés de façon
centralisée à l'aide d'un \emph{DSL}, permettant une configuration plus
souple qu'un simple système clef-valeur.

\section{Contraintes non fonctionnelles}

Les autres contraintes, non fonctionnelles, régissant les choix
architecturaux sont les suivantes.

Onitu devra être conforme au protocole définit par \emph{Ubuntu One},
puisque compatible avec le client officiel.

L'utilisation de certaines bibliothèques, telles que \emph{Twisted}, non
portée en Python3, contraint à l'utilisation d'une version antérieure.

Dans un soucis de simplicité, le \emph{DSL} proposé devra être
facilement compréhensible par l'utilisateur, ne devra pas le rebuter. Ce
dernier devrait sans problème pouvoir bénéficier d'une configuration
fine de son système.

C'est principalement dans un soucis de protection et contrôle des
données que la solution Onitu sera utilisée, c'est pourquoi sa sécurité
doit être maximale.
