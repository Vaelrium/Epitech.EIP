\section{Points sensibles}
Onitu doit être réalisé dans un délais de deux ans et demi. Pendant cette période, plusieurs points sensibles sont à surveiller.\\

Un premier point sensible est le protocole d'Ubuntu One. Onitu n'est pas à l'abri de devoir subir des changements majeurs si le protocole venait à évoluer. Cependant, une telle évolution n'est pour le moment pas prévue par Canonical\\

La portabilité est un autre point sensible. En effet, cela entraîne de nombreux tests sur un maximum de plateformes possibles, et le risque de devoir se passer d'une bibliothèque ou d'une fonctionnalité car elle n'est pas supportée partout. Si une plateforme pose trop de problème et n'est pas utilisée par beaucoup d'utilisateurs, l'abandon de son support est à envisager.\\

Enfin, Onitu est confronté au théorème CAP, s'appliquant aux systèmes de calculs distribués, et affirmant que trois contraintes ne peuvent simultanément être respectées: la cohérence, la disponibilité, et la résistance au morcellement. Onitu a essentiellement pour vocation de préserver la cohérence et la disponibilité.\\

De manière générale, le développement d'un serveur est toujours un sujet délicat à ne pas prendre à la légère. De nombreux tests de différentes catégories doivent être effectués réguliérement, et les retours des utilisateurs doivent être écoutés.