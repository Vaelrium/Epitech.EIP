\documentclass[11pt]{report}

\usepackage[utf8]{inputenc}
\usepackage[french]{babel}
\usepackage{fullpage}
\usepackage{graphicx}
\usepackage{fancyhdr}	% headers/footers
\usepackage{xcolor}		% to use our own color
\usepackage{lastpage}	% to easily know the total number of pages
\usepackage{titling}	% to easily know the total number of pages
\usepackage{colortbl}	% to put color in a table background
\usepackage{datetime}	% to allow us set a new date formatting
\usepackage{multirow}   % to allow multirows in tables
\usepackage[colorlinks,linkcolor=black]{hyperref}
\usepackage{palatino}
%% \usepackage[colorlinks=false, urlcolor=blue, breaklinks, pagebackref, citebordercolor={0 0 0}, filebordercolor={0 0 0}, linkbordercolor={0 0 0}, pagebordercolor={0 0 0},
%%                      runbordercolor={0 0 0}, urlbordercolor={0 0 0}, pdfborder={0 0 0}]{hyperref}

% Custom defines zone

% Define useful hand-made colors
\definecolor{epiBlue}{RGB}{0,110,255}
\definecolor{lightGray}{gray}{0.92}

% Bit of code to bold an entire table row
% http://tex.stackexchange.com/questions/4811/make-first-row-of-table-all-bold
\newcolumntype{$}{>{\global\let\currentrowstyle\relax}}
\newcolumntype{^}{>{\currentrowstyle}}
\newcommand{\rowstyle}[1]{\gdef\currentrowstyle{#1}%
  #1\ignorespaces
}

% Defining a "dd/mm/yyyy" date format
\newdateformat{dashDate}{\twodigit{\THEDAY}/\twodigit{\THEMONTH}/\twodigit{\THEYEAR}}

% Define Document Title
\newcommand{\ProjectTitle}{Onitu}
\newcommand{\DocTitle}{EIP Onitu}
\newcommand{\SubTitle}{Cahier des Charges 2 (CDC2)}

% Defining some logo image names
\newcommand{\ProjectLogo}{logo_onitu}
\newcommand{\EIPLogo}{logo_eip.png}
% Setting the space between each page's header and its content
\setlength{\headsep}{0.2in}

% end of Defines

% fancyhdr-specific commands
\setlength{\headheight}{15.2pt}

%% Defining headers and footers contents.

% Big dirty hack of the "empty" pagestyle to show header and footer on the title page (in wait of a better solution)
\fancypagestyle{empty}
{
	\renewcommand{\headrulewidth}{0pt}
	\renewcommand{\footrulewidth}{1pt}
	\fancyhead[L]{\includegraphics[height=35pt]{\EIPLogo}}
	\fancyhead[C]{}
	\fancyhead[R]{\includegraphics[height=35pt]{\ProjectLogo}}

	\fancyfoot[L]{}
	\fancyfoot[C]{\textcolor{epiBlue}{\textbf{\underline{\DocTitle\ — \SubTitle}}}}
	\fancyfoot[R]{}
}

\fancypagestyle{EIP}
{
	\renewcommand{\headrulewidth}{0pt}
	\renewcommand{\footrulewidth}{1pt}
	\fancyhead[L]{\includegraphics[height=35pt]{\EIPLogo}}
	\fancyhead[C]{}
	\fancyhead[R]{\includegraphics[height=35pt]{\ProjectLogo}}

	\fancyfoot[L]{\textbf{\underline{\scriptsize{\leftmark}}}}
	\fancyfoot[C]{}
	\fancyfoot[R]{
		\thepage/\pageref{LastPage}
	}
}

\pagestyle{EIP} % does not seem to work ...

% end of fancyhdr stuff

%Gummi|063|=)

%\title{The Title\\\normalsize A Sub-title}
\title{
	\huge{\textbf{\textcolor{epiBlue}{\DocTitle} } }\\
	\Large{\textbf{\emph{\textcolor{gray}{\SubTitle} } } }
}


\begin{document}
%\fontfamily{cm}\selectfont
\addtocontents{toc}{\protect\refstepcounter{page}} % makes the table of contents count pages from 1 (one)
\maketitle

\thispagestyle{empty}
\vspace*{10mm}

\textbf{\emph{\textcolor{onitu}{\large{Résumé du document} } } }\\

Résumé !!!

\clearpage


\thispagestyle{empty}
\vspace*{10mm}
\textbf{\emph{\textcolor{epiBlue}{\large{Description du document} } } } \\

\vspace*{2mm}

\begin{tabular}{|>{\columncolor{epiBlue} \color{lightGray} \bfseries } l|l|}
\hline
	Titre & \DocTitle\\
\hline
	Date & \dashDate\today \\
\hline
	Auteur & Alexandre Baron\\
\hline
	Responsable & Louis Roché\\
\hline
	E-Mail & onitu\_2015@labeip.epitech.eu\\
\hline
	Sujet & CDC2\\
\hline
	Mots clés & Cahier, Charges, Technique, Hippocampe\\
\hline
	Version du modèle & 2.1\\
\hline
\end{tabular}

\vspace*{10mm}

\textbf{\emph{\textcolor{epiBlue}{\large{Tableau des révisions} } } }\\

\vspace*{2mm}

\begin{tabular}{|$l|p{4cm}|p{2cm}|p{5cm}|}
\hline
\rowcolor{epiBlue}
\rowstyle{ \color{lightGray} \bfseries}
	Date & \textcolor{lightGray}{\textbf{Auteur}} & \textcolor{lightGray}{\textbf{Section(s)}} & \textcolor{lightGray}{\textbf{Commentaires}}\\
\hline
	10/04/2013 & Alexandre Baron & Toutes & Nouveau template \\
\hline
	03/05/2013 & Alexandre Baron & Toutes & Lifting du template: Ajout de l'icône, changement des couleurs\\
\hline
	04/05/2013 & Alexandre Baron & Headers, Footers & Mis l'icone avec "Onitu", remonté l'icone EIP du footer au header \\
\hline
	05/05/2013 & Antoine Rozo et Yannick Peroux & Toutes & Corrections des fautes d'orthographe et reformulations \\
\hline
	06/05/2013 & Louis Roché & Titre & Corrections du titre \\
\hline
	15/05/2013 & Alexandre Baron & Template & Retiré l'icône, mis les titres en bas à gauche\\
\hline
	30/05/2014 & Alexandre Baron & Document & Relecture, corrections et mises à jour\\
\hline
	01/07/2014 & Louis Roché & Document & Ajouts pour arrivée de maurin_t\\


\hline
\end{tabular}

\tableofcontents
\addtocontents{toc}{\protect\thispagestyle{empty}
                    \protect\pagestyle{empty}}
\thispagestyle{empty}

\chapter{Rappel de l'EIP}
\thispagestyle{EIP} % seems mandatory
\setcounter{page}{1} %reset the page count

\section{Qu'est-ce qu'un EIP et Epitech}
Epitech, école d'expertise informatique en cinq ans, offre aux étudiants l'opportunité de réaliser un projet de fin d'études sur trois ans, l'EIP (pour \emph{Epitech Innovative Project}).\\

À ce titre, les élèves doivent s'organiser en un groupe d'au moins cinq personnes et choisir un sujet porteur de nouveautés ou améliorant un ancien sujet. L'EIP est un passage obligatoire et unique dans la scolarité de l'étudiant, de par son envergure (18 mois) et la préparation requise. Le but est, à la fin du temps imparti, d'obtenir un projet commercialisable.


\section{Principe de base du système futur}
    Onitu est un projet visant à proposer une implémentation libre et Open Source du serveur d’Ubuntu One.\\

    Ubuntu One est un service de Canonical (sponsor officiel d'Ubuntu) permettant de disposer d’un espace de stockage en ligne qui sera synchronisé entre différents ordinateurs et périphériques compatibles via un logiciel client. Le client et le protocole d’Ubuntu One sont disponibles sous licence libre. Néanmoins, le serveur est propriétaire et n’a pas été publié.\\

    L'objectif d'Onitu de proposer un équivalent libre à ce serveur, afin de profiter des fonctionnalités d’Ubuntu One tout en maîtrisant le stockage des données et des informations.\\

    Les fichiers gérés par Onitu pourront être stockés sur un serveur administré par un utilisateur, ou bien sur des services tiers comme Dropbox, Amazon S3, ou Google Drive.\\

    La cible première d'Onitu est l'utilisateur averti, soucieux des problématiques de centralisation des données, et son entourage, à qui il fera profiter le serveur mis en place. Il n'est pas forcément technicien mais assez curieux, le profil même de l'utilisateur Ubuntu.
    Onitu vise aussi à être utilisé au sein d'entreprises ayant la volonté de maîtriser facilement le stockage de leur données.\\


\chapter{Présentation de l'environnement de réalisation}
\thispagestyle{EIP} % seems mandatory

\section{Environnement de réalisation}
Le logiciel de gestion de version Git est utilisé, afin de profiter d'un système de branches efficace et évitant ainsi les conflits dans le code.
Les dépôts Git sont hébergés sur le site Github, offrant des fonctionnalités annexes telles que les \textit{issues} ou le \textit{Wiki}, proposant ainsi un espace clair pour discuter de différents points du projet, gérer les contribution de la communauté et rédiger la documentation.\\

Aucun environement de développement n'est imposé aux personnes souhaitant contribuer. Il est donc important que les dépôts restent vides de tous fichiers liés aux éditeurs de textes.\\

La majeure partie du projet est réalisée avec le langage Python, qui est interprété, dynamiquement typé, et reconnu pour sa flexibilité et sa robustesse.\\
Tout le code Python produit doit être en respect avec la PEP8, imposant diverses règles qui amènent à un code propre et agréable à lire, et qui fait office de convention au sein des développeurs.\\

Dans la mesure du possible, le serveur doit être portable sur une majorité de plateformes. Ce travail est facilité par le langage Python, mais nécessite une attention particulière.

\section{Environnement matériel}
Plusieurs machines seront nécessaires pour tester les différents composants. Ces machines peuvent être virtualisées, car elles nécessiterons peut de ressources dans la plupart des cas.\\

La majorité du temps, les tests peuvent se dérouler localement et ne nécessitent donc pas de structure externe. Cepandent, réguliérement, des essais devront être effectués à échelle réelle, nécessitant plusieurs machines sur des réseaux différents.\\

Les contraintes liés à la portablité entraînent de faire des tests sur différentes plateformes. Des machines fonctionnant sous ces différentes plateformes sont donc requises.\\

Un serveur d'intégration doit être utilisé afin de faire tourner les tests de premier niveaux réguliérement.\\

\section{Architecture technique}

\section{Existing components}
\thispagestyle{EIP} % seems mandatory
Several existing clients use the Ubuntu One API, on several platforms. Onitu is compatible with those clients on the sole condition they allow the user to specify the server's address.\\

The Ubuntu One protocol, \textit{ubuntuone-storage-protocol}, uses Google's data organisation solution \textbf{Protocol Buffers}. Onitu uses the same protocol and the same tools.\\

The network comunication will mostly be using \textbf{Twisted}, a Python framework for \emph{event-driven} network programing. This framework is largely used and is in actif development.\\

Several libraries can be used to comunicate with the different services, through the \textit{drivers}. The choice of those libraries is left to the developers of the \textit{drivers}. However it must be reasonable and not cause any portability issues.

\section{Gestion de la sécurité}

La gestion de la sécurité est présente à plusieurs endroits dans le cycle de vie d'un logiciel. 

La première chose qui vient à l'esprit est de sécuriser le logiciel produit. C'est souvent par des attaques lorsque le logiciel est en production que celui-ci est compromis. Mais ce n'est pas le seul risque. En effet, il faut prêter la même attention au procesus de développement lui-même, la gestion des différentes ressources qui lui sont liées, et à la façon dont le projet va être maintenu.

\subsection{Infrastructure \& maintenance}

Le processus de développement s'organise autour de Git et en particulier Github. Les membres du groupe d'EIP ont un accès administrateur sur cette plateforme. Cela leur permet automatiquement de contribuer au code et d'utiliser les différents outils fournis par Github. Git permet d'avoir une traçabilité des modifications du code car chaque modification est signée et authentifiée.\\

Le code source est hébergé sur les serveurs de Github, mais chaque développeur possède une copie locale du dépôt à tout instant, donc une perte de données de la part de l'hébergeur n'est aucunement dramatique. Il n'y a pas non plus de problème de confidentialité, car le projet est opensource, le code source n'est donc pas privé.\\

Tous les dépôts d'Onitu sont ouverts aux contributions extérieures, via des \textit{pull-requests}. Il s'agit d'une demande d'intégration au dépôt principal de code développé sur un autre dépôt par une personne ne faisant pas partie des contributeurs habituels. Si une \textit{pull-request} est faite, le code externe sera audité pour s'assurer de sa qualité, mais aussi pour vérifier que le code n'est pas malveillant, n'apporte pas de vulnérabilités et n'enlève rien à l'efficacité des solutions déjà en place.\\

Pour ce qui est de la maintenance et de la gestion des retours utilisateurs, ceux-ci pourront se faire via le système de tickets fourni par Github. Tout le monde peut créer des tickets, mais ceux-ci ne peuvent être clos, supprimés ou administrés que par les membres du groupe d'EIP.

\subsection{Processus de développement}

La quasi-totalité du projet est développée en Python. Cela élimine une grande partie des vulnérabilités classiques que l'on rencontre dans des projets développés avec des langages plus bas niveau, notamment concernant la gestion de la mémoire qui n'est plus à la charge du développeur. De plus, Python lève des exceptions et arrête le programme dès qu'une valeur n'est pas correctement utilisée (\textit{overflows}, \textit{casts} incorects…).\\

Les interactions avec les bases de données et les différents protocoles sont faites au travers de bibliothèques proposant un usage simple et moins propice aux erreurs qu'un accès direct aux ressources.\\

Cependant, un certain nombre de vulnérabilités ne peuvent être évitées que par du bon sens et un respect des meilleures pratiques. C'est notamment le cas pour tout ce qui concerne les fonctions qui permettent d'évaluer du code, ou d'accéder aux ressources système. L'appel à ces fonctions devra donc être particulierement surveillé et justifié, plus encore lorsqu'elles peuvent être influencées par des entrées utilisateurs.\\

Des phases d'audit seront organisées, où l'ensemble du code sera vérifié par des développeurs n'ayant pas travaillé sur ces parties, de façon à avoir un deuxième avis objectif sur la question de la sécurité. De plus, nous avons un partenariat avec le Laboratoire de Sécurité d'Epitech Toulouse qui pourra effectuer un ou plusieurs audits extérieurs sur le code et les différents protocoles utilisés.

\subsection{Sécurité dans Onitu}

En plus des différentes mesures expliquées ci-dessus, une partie de la sécurité d'Onitu est apportée par ses composants principaux et les protocoles qui y seront utilisés.

Pour authentifier les clients, le protocole \textit{OAuth} est utilisé. Il s'agit d'un système d'authentification par \textit{token} qui permet de gérer les accès parallèles depuis differents ordinateurs, et éventuelement leur révocation.\\

Pour les connexions entre client et serveur, nous utilisons le module \textbf{\emph{curve}} de ZeroMQ (ØMQ) afin de sécuriser les communications.\\

Les fichiers stockés ne sont pas chiffrés, car le serveur nécessitera la version originale lors d'une demande de téléchargement. Cependant, rien n'empêche l'utilisateur de chiffrer ses fichiers en local avant de les synchroniser avec le serveur Onitu.
\subsubsection{Drivers}
En ce qui concerne les différents modules nécessaires à la gestion du stockage de fichiers sur le serveur Onitu, la sécurité de leurs protocoles respectifs ne dépend pas de nous. Nous ne pourrons que conseiller l'utilisateur sur la configuration de la solution en fonction de ses attentes, (confidentialité, intégrité, authentification…), par exemple en lui conseillant \textit{SFTP} au lieu de \textit{FTP}.

\section{Points sensibles}
\thispagestyle{EIP} % seems mandatory

Un élément essentiel est le protocole d'Ubuntu One sur lequel nous nous basons. Si Canonical venait à le changer brusquement, cela constituerait un point sensible assez important. Il est peu probable que cela se produise, mais nous aurions alors le choix entre deux solutions: si les changements sont mineurs ou que notre implémentation est assez peu complète, nous pourrions nous adapter à ce nouveau protocole. Dans le cas contraire, libre oblige, nous partirions sur un fork du projet en gardant ainsi l'ancien protocole chez les clients.\\

Un autre point est la possibilité que d'autres serveurs concurrents se montent pendant le développement de notre projet.\\

On peut également se poser la question de la compatibilité multi-OS (Windows, Mac OS X, *BSD, GNU/Linux). Ce point est important car il va nous orienter dans le choix du langage et des bibliothèques utilisé ainsi que des contraintes matériels, par exemple, avoir un mac pour les tests sur Mac os.\\

Enfin, nous serons confrontés au théorème CAP, s'appliquant aux systèmes de calcul distribué, et affirmant que trois contraintes ne peuvent simultanément être respéctées: la cohérence, la disponibilité, et la résistance au morcellement.


\chapter{Description des différentes parties du programme à réaliser}
\thispagestyle{EIP} % seems mandatory
\section{Architecture générale}
Le but du projet est de permettre à des applications de modifier des fichiers stockés à différents endroits, via le protocole défini par UbuntuOne.
Pour cela, plusieurs composants sont nécessaires :
\begin{itemize} 
   \item Le serveur : Le serveur reçoit des instructions via l'API d'UbuntuOne, par HTTP.
   \item Les clients : Les clients sont la partie utilisateur. Ces clients échangent avec le serveur via l'API. 
   \item Les drivers : Les drivers permettent d'échanger entre le serveur, et les différentes solutions de stockage. 
\end{itemize}

\section{Serveur}
Le serveur est la partie centrale du projet. Il doit interpréter les instructions envoyés par les clients, et y répondre.

Une implémentation a été effectuée par l'entreprise Canonical, à l'origine d'UbuntuOne. Cette implémentation utilise les services de Cloud Computing EC2 et S3 d'Amazon. Suite à un partenariat entre ces deux entreprises, le code source du serveur n'est pas disponible, et ne le sera à priori jamais.

Notre serveur utilisera une base de données pour stocker les méta-données des fichiers.

Onitu permet aux utilisateurs de choisirs plusieurs moyens pour stocker leur données. Du point de vue du serveur, cela se fait grâce aux méta-données stockées en base de données, qui contiennent les informations nécessaires aux drivers pour récupérer les données. 

\section{Drivers}
Les drivers permettent la communication entre le serveur et les différentes solutions de stockage. Ils sont construits autour d'une interface commune, qu'ils étendent afin d'intéragir avec la solution à laquelle ils sont destinés.

Tous les drivers doivent permettre d'effectuer les quatre opérations de base : la création d'un fichier, sa récupération, son édition et sa suppression (connu sous l'accronyme CRUD, pour Create, Read, Update, Delete).

Du fait du côté ouvert d'Onitu, n'importe qui peut créer son propre driver. Ceci dit, plusieurs sont prévus et seront maintenus par l'équipe de développement :
\begin{itemize}
    \item Stockage local : Pour permettre de stocker des fichiers sur le serveur où est installé Onitu.
    \item Dropbox/Box.com/Google Drive/SkyDrive : Onitu pourra utiliser ces services tiers pour stocker les fichiers, si l'utilisateur possède un compte.
    \item (S)FTP/SSH/NFS/Webdav/Samba/HTTP : Des drivers seront dévelopés afin d'être compatible avec n'importe quel serveur supportant au moins un de ces protocoles.
    \item UbuntuOne : Il sera possible d'utiliser n'importe quel serveur compatible avec l'API d'UbuntuOne comme serveur de stockage.
\end{itemize}

\section{Clients}

\section{Interface Web}


\chapter{Description des tests de premier niveau}
\thispagestyle{EIP} % seems mandatory



\chapter{Description de la stratégie marketing et communication}
\thispagestyle{EIP} % seems mandatory
\section{Communication et marketing}
Les utilisateurs avertis, sensibles à la façon dont leurs données sont traitées, sont la principale cible d'Onitu. Onitu vise également les entreprises, elles aussi sensibles à la façon dont leurs données sont stockées. Ce n'est pas un public de masse, et la façon de communiquer doit être choisie en conséquence.

\subsection{Communication}
Le support de communication principal d'Onitu est Internet, et plus précisement les sites spécialisés, parlant des nouvelles technologies. Il faudra convaincre ces sites d'écrire des articles sur le projet afin de le faire connaître.\\


\subsection{Marketing}
D'un point de vue référencement, le nom Onitu a été choisi car aucun autre projet ne porte ce nom, et le référencement sur les moteurs de recherche ne devrait donc pas poser de problèmes.\\

La charte graphique du projet est définie. Elle décrit notamment les couleurs associées au projet, la police (Monaco), ainsi que le logo du projet.\\

Un site web vitrine a d'ores et déjà été mis en place afin de décrire le projet, de proposer un guide utilisateur, ainsi que des liens vers une documentation plus détaillée.


\chapter{Organisation projet}
\thispagestyle{EIP} % seems mandatory
Nous avons la chance de débuter le projet dans des conditions proches de ce que nous vivrons l'année prochaine, au décalage horaire près, dans le sens où nous sommes répartis sur trois sites différents. Ainsi, notre organisation prend en compte cet élément dès le départ, et nous ne subirons pas de choc, ou moindre, l'an prochain.\\

Pour cela, nous avons au début du projet mis en place un système de réunions hebdomadaires, à savoir chaque mercredi soir, où nous discutons principalement des documents à rendre ainsi que de leurs échéances. En période de crise, il nous arrive aussi de prévoir de nouvelles réunions.\\

En parallèle, nous sommes constamment présents sur un canal IRC sur lequel nous débattons des axes que nous souhaitons prendre pour notre projet. Il nous sert aussi aux préparatifs des réunions, nous y fixons en effet les heures et nous réunissons préalablement, après quoi nous passons sur le système de bulles de Google pour procéder à la réunion proprement dite.\\

De plus, notre projet étant hébergé sur Github, nous bénéficions des avantages de ce dernier dans sa gestion, à savoir principalement un \textit{wiki} sur lequel nous écrivons par exemple nos compte-rendus de réunions ou mettons à disposition des conversations importantes que nous avons pu avoir, ainsi qu'un système d'\textit{issues} et \textit{milestones}, qui nous servent à la répartition des tâches, et à avoir une vision des échéances.

\chapter{Annexes}
\thispagestyle{EIP} % seems mandatory
\begin{itemize}
\renewcommand{\labelitemi}{$\bullet$}
\item \href{http://eip.epitech.eu/2015/onitu/}{Le site vitrine} d'Onitu.
\item \href{http://onitu.readthedocs.org/en/latest/index.html}{La documentation technique} d'Onitu.
\item \href{https://github.com/onitu/}{Le Github} de l'organisation Onitu.
\item \href{http://www.python.org/dev/peps/pep-0008/}{La PEP8} qui définie une
norme pour le code Python.\\

\end{itemize}


\end{document}
