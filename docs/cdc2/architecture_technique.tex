\section{Architecture technique}
Pour gérer le projet, l'ensemble du travail est regroupé sur une organisation Github : \href{https://github.com/onitu/}{Onitu}.\\

Cette organisation permet notamment d'avoir un dépôt Git pour chacune des parties du projet. Chaque dépôt dispose d'une solution de suivi des problèmes, ce qui permet de surveiller l'avancement du projet et discuter des problèmes.
Pour toutes les étapes du développement, des \textit{milestones} sont créées et des \textit{issues} y sont liées. Cela permet de suivre le bon avancement du travail et de regrouper tout ce qui est lié.\\

Chaque dépôt peut aussi disposer d'un wiki, ce qui permet de mettre à disposition de tous la documentation de chacune des parties, que ce soit la documentation technique à destination de développeurs ou de la documentation utilisateur pour installer ou configurer le logiciel.\\

Toutes les parties du projet sont donc développées séparément les unes des autres. Les connexions entre elles se font grâce aux \textit{submodules} de Git, qui permettent d'intégrer un dépôt Git à l'intérieur d'un autre.\\

Tous les membres du groupe d'EIP ont les droits d'édition sur l'ensemble des dépôts de l'organisation Onitu. Ils peuvent donc participer à chacune des parties du projet, aussi bien sur du développement que sur de la documentation ou des rapports de bogues.\\

L'outil de revue de code intégré à Github est utilisé pour faire valider chaque avancement, dans une optique de qualité et de sécurité.
