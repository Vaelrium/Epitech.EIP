\section{Gestion de la sécurité}

La gestion de la sécurité est présente à plusieurs endroits dans le cycle de vie d'un logiciel. La première chose qui vient à l'esprit est de sécuriser le logiciel produit. C'est souvent par des attaques lorsque le logiciel est en production que celui-ci est compromis. Mais ce n'est pas le seul risque. En effet il faut prêter la même attention au procesus de développement lui-même, la gestion des différentes ressources qui lui sont liées et à la facon dont le projet va être maintenu.

\subsection{Infrastructure \& maintenance}

Le processus de développement s'organise autour de Git et en particulier Github. Les membres du groupe d'EIP ont un accès administrateur sur cette plateforme. Cela leur permet automatiquement de contribuer au code et d'utiliser les différents outils fournis par Github. Git permet d'avoir une traçabilité des modifications du code car chaque modification est signée et authentifiée.\\

Le code source est hébergé sur les serveurs de Github, mais chaque développeur possède une copie du dépot en local à tout instant, donc une perte de données de la part de l'hébergeur n'est aucunement dramatique. Il n'y a pas non plus de problème de confidentialité, car le projet est opensource et le code source n'est donc pas privé.\\

Tous les dépôts de d'Onitu sont ouverts aux contributions extérieures, via des \textit{pull-request}. Il s'agit d'une demande d'intégration au dépôt principal de code développé sur un autre dépôt par une personne ne faisant pas partie des contributeurs habituels. Si un \textit{pull-request} est fait, le code externe sera audité pour s'assurer de sa qualité, mais aussi pour vérifier que le code n'est pas malveillant, n'apporte pas de vulnérabilités et n'enleve rien à l'efficacité des solutions déjà en place.\\

Pour ce qui est de la maintenance et de la gestion des retours utilisateurs, ceux-ci pourront se faire via le système de tickets fourni par Github. Tout le monde peut créer des tickets mais ceux-ci ne peuvent être clos, supprimés ou administrés que par les membres du groupe d'EIP.

\subsection{Processus de développement}

La quasi-totalité du projet est développée en Python. Cela élimine une grande partie des vulnérabilités classiques que l'on rencontre dans des projets développés avec des langages plus bas niveau, notamment concernant la gestion de la mémoire qui n'est plus à la charge du développeur. De plus, Python lève des exceptions et arrête le programme dès qu'une valeur n'est pas correctement utilisée. (\textit{overflows}, \textit{casts} incorects…)\\

Les interactions avec les bases de données et les différents protocoles sont faites au travers de librairies proposant un usage simple et moins propice aux erreurs qu'un accès direct aux ressources.\\

Cependant, un certain nombre de vulnérabilités ne peuvent être évitées que par du bon sens et un respect des meilleures pratiques. C'est notamment le cas pour tout ce qui concerne les fonctions qui permettent d'évaluer du code, ou d'accéder aux ressources système. L'appel à ces fonctions devra donc être particulierement surveillé et justifié, plus encore lorsqu'ils peuvent être influencés par des entrées utilisateurs.\\

Des phases d'audit seront organisées, où l'ensemble du code sera vérifié par des développeurs n'ayant pas travaillé sur ces parties, de façon à avoir un deuxième avis objectif sur la question de la sécurité. De plus, nous avons un partenariat avec le Laboratoire de Sécurité d'Epitech Toulouse qui pourra effectuer un ou plusieurs audits extérieurs sur le code et les différents protocoles utilisés.

\subsection{Sécurité dans Onitu}

En plus des différentes mesures expliquées ci-dessus, une partie de la sécurité d'Onitu est apportée par ses composants principaux et les protocoles qui y seront utilisés.

\subsubsection{Ubuntu One}
Du fait de sa compatibilité avec Ubuntu One, Onitu ne dispose pas de marge de manœuvre sur le protocole. Cependant, celui-ci a été conçu et implémenté par des développeurs professionels et est utilisé en production depuis plusieurs années.\\

Comme Ubuntu One, Onitu utilise \textit{SSL} (Secure Socket Layer) et des certificats numériques pour assurer l'authentification du serveur et le chiffrement des connexions.\\

Pour authentifier les clients, le protocole \textit{OAuth} est utilisé. Il s'agit d'un système d'authentification par \textit{token} qui permet de gérer les accès parallèles depuis differents ordinateurs, et éventuelement leur révocation.\\

Les fichiers stockés ne sont pas chiffrés, car le serveur nécessitera la version originale lors d'une demande de téléchargement. Cependant, rien n'empêche l'utilisateur de chiffrer ses fichiers en local avant de les synchroniser avec le serveur Onitu.

\subsubsection{Backend drivers}
En ce qui concerne les différents modules nécessaires à la gestion du stockage de fichiers sur le serveur Onitu, la sécurité de leurs protocoles respectifs ne dépend pas de nous. Nous ne pourrons que conseiller l'utilisateur sur la configuration de la solution en fonction de ses attentes, (confidentialité, intégrité, authentification…), par exemple en lui conseillant \textit{SFTP} au lieu de \textit{FTP}.
