Nous avons la chance de débuter le projet dans des conditions proches de ce que nous vivrons l'année prochaine, au décalage horaire près, dans le sens où nous sommes répartis sur trois sites différents. Ainsi, notre organisation prend en compte cet élément dès le départ, et nous ne subirons pas de choc, ou moindre, l'an prochain.\\

Pour cela, nous avons au début du projet mis en place un système de réunions hebdomadaires, à savoir chaque mercredi soir, où nous discutons principalement des documents à rendre ainsi que de leurs échéances. En période de crise, il nous arrive aussi de prévoir de nouvelles réunions.\\

En parallèle, nous sommes constamment présents sur un canal IRC sur lequel nous débattons des axes que nous souhaitons prendre pour notre projet. Il nous sert aussi aux préparatifs des réunions, c'est en effet là que nous fixons les heures et nous réunissons préalablement, après quoi nous passons sur le système de bulles de Google pour procéder à la réunion proprement dite.\\

De plus, notre projet étant hébergé sur Github, nous bénéficions des avantages de ce dernier dans sa gestion, à savoir principalement un \textit{wiki} sur lequel nous écrivons par exemple nos compte-rendus de réunions ou mettons à disposition des conversations importantes que nous avons pu avoir, ainsi qu'un système d'\textit{issues} et \textit{milestones}, qui nous servent à la répartition des tâches, et à avoir une vision des échéances.