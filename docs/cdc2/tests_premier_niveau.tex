Pour s'assurer du bon fonctionnement du projet tout au long du développement et pour faciliter les possibles évolutions, des tests sont nécessaires.\\

Des tests sur chacune des parties du projet doivent donc être créés au fur et à mesure que du code sera écrit. Il est notamment nécessaire de tester :\\

\begin{itemize}
\renewcommand{\labelitemi}{$\bullet$}
	\item L'intégrité des fichiers après synchronisation entre deux services différents;
	\item La portabilité, le bon fonctionnement de l'application autant en environnements Unix que sur Windows;
	\item Les régressions introduites par de nouveaux changements: nous devons nous assurer à chaque modification que l'ensemble du code reste cohérent et qu'elle n'engendre pas de nouveaux bogues;
	\item La rapidité: au moyen de \emph{benchmarks}, il est nécessaire d'étudier les performances de l'application afin de cibler de possibles goulots d'étranglement et conserver un logiciel le plus rapide possible.
\end{itemize}

\vspace{0.5cm}

Pour faciliter ces tests et être certain qu'ils sont effectués de manière régulière, ils seront effectués avec un outil d'intégration continue à chaque \textit{commit}. Les deux outils suivants peuvent être utilisés :\\

\begin{itemize}
\renewcommand{\labelitemi}{$\bullet$}
\item \href{https://travis-ci.org}{travis} qui est un service web lié à Github ;
\item \href{http://jenkins-ci.org}{jenkins} qui est un logiciel libre à installer sur un serveur, mais qui permet une beaucoup plus grande variété de tests que \textit{Travis}.
\end{itemize}
