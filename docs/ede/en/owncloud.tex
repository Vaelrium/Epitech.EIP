\section{Owncloud}
\thispagestyle{EIP} % seems mandatory
\subsection{Presentation}
OwnCloud is an online web based hosting application. It is open-sourced using the AGPL license and can be install on any server disposing of PHP and SQL.\\
It is a solution the user has to install himself and those not provide any storage space.\\

\subsection{History}
Anounced during the KDE Camp in 2010, the project has evolved since. The development is very actif and is folowed by the comunity. (More than 10000 commits and 1500 bug reports in less than 3 years).\\

In 2011, a compagny formed around the project and proposed more advanced services targeted at buisnesses. At the end of 2012, the compagny raised 2.5 million dolars for the project.

\subsection{Description}
OwnCloud is based on a collection of applications. It is a modular system that enjoys a lot of functions. (File editing, music streaming, calender and contact syncronization, photo galeries, ...).\\
Several clients are available on most platforms and a web-based client exists.\\

\subsection{Critics}
A lot of users complain about the amount of bugs in OwnCloud and about the quality of its code. Its bigest disadvantage is the fact it is very easy to install but that it shows weaknesses in the long run.
For file sharing OwnCloud is based on the WebDAV technology which strugles with big files.
Since the version 4.5 it is possible to interface OwnCloud with google Frive and Dropbox. This, still experimental, feature is the most close to Cloud Computing. As a matter of fact OwnCloud is not able to distribute the charge on several servers which is mandatory in cloud computing.
