\section{Risques}
Notre projet étant situé dans le domaine du cloud, très à la mode ces dernières années, et qui voit régulièrement naître de nouvelles solutions, il y a un risque qu'une solution équivalente et concurrente, voire plus avancée, voit le jour, mais nous supposons pour l'instant que ce ne sera pas le cas.

\section{Hypothèses}
Nous suposons que les developpeurs travaillant déjà sur les différents systèmes par lesquels notre projet est concerné seront ouverts et disposés à nous aiguiller. D'après nos premiers échanges, nous supposons aussi qu'ils ne sont pas hostiles à notre implémentation open-source.

Nous avons l'intention de ne développer qu'un serveur et donc de nous reposer sur des clients ubuntu-one déjà existants pour nos tests. Ces clients devront très probablement être modifiés légèrement de façon à pouvoir les faire fonctionner avec d'autres serveurs que ceux fournis par canonical. Nous supposons que ces modification seront simples et bien reçues par leurs auteurs respectifs.

Nous espérons pouvoir nous organiser de façon efficace durant la quatrième année en faisant des groupes de travail par fuseau horaire, ce qui devrait augmenter notre productivité tout en nous permettant une bonne avancée du projet durant l'année à venir.

\section{Contraintes}
Le projet étant open-source, nous devrons respecter certaines contraintes imposées par les licenses choisies. Celles-ci doivent encore être étudiées et déterminées.

L'architecture complète du projet devra être fixée avant juin 2013 pour pouvoir être presentée durant la soutenance avec le LabEIP.

Une des contraintes principales de notre projet sera inhérente au développement de celui-ci. En effet, il faudra tenir compte du besoin de modification des clients et de l'implémentation d'au moins certaines interfaces de stockage (par exemple, le stockage local) avant de pouvoir réellement tester nos idées dans des conditions réelles.

Nous devrons aussi être capables d'organiser des réunions régulières, et ce malgré le décalage horaire.

Enfin, dû à l'éloigenement géographique, il sera plus compliqué de maintenir une plate-forme de test pour notre solution d'hébergement. Nous serons probablement ammenés à prendre un serveur dédié de test accessible depuis nos différentes destinations en tek4.

