\section{What is an EIP at Epitech}
Epitech, the European Institute for Technology, is a school in five years. It offers a final major project starting during the student's third year. This is called an EIP: \emph{Epitech Innovative Project}.


The students must organize and form a group of at least five people. They must choose a project that brings new ideas or improves uppon an older project. The EIP is a mandatory and unique passage in an Epitech's student's life because it is so long (months) and the amount of preparation that is required. The goal is to have a marketable product in the end.


\section{Basic principle of the future system}
Onitu is a project aiming to deliver a free, open source implementation of the Ubuntu One server. Ubuntu One is a Canonical (the official Ubuntu sponsor) service permitting to dispose of an online storage space synchronized between multiple computers and compatible devices through a client software. The client and protocol of Ubuntu One are available under free license. However, the server is proprietary and has not been published.


Our goal is to propose a free equivalent of that server: Onitu. It will allow to easily use other hosting services, "in the cloud" or not, in order to extend the available space. For example, the user will be able to use his Dropbox or Amazon S3 account, or FTP servers to store his files, and then synchronize it via Ubuntu One.


The main goal of Onitu is the advanced user, who cares about data centralization problems, and his near circle, to whom he'll make profit the so established server. He doesn't have to be a technical expert, yet rather curious, typically the Ubuntu user profile. Onitu also aims to be used in enterprises who want to easily master the storage of their data.

